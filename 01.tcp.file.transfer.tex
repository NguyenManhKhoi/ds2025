\documentclass[a4paper,12pt]{article}

% Packages for formatting and better output
\usepackage[utf8]{inputenc}
\usepackage{amsmath}
\usepackage{graphicx}
\usepackage{hyperref}
\usepackage{listings}
\usepackage{color}

% Set up page margins
\usepackage[margin=1in]{geometry}

\title{File Transfer over TCP/IP using Sockets\\ \large Lab Report}
\author{Nguyen Manh Khoi \\ 22BI13219}
\date{\today}

\begin{document}

\maketitle

\section{Introduction}
File transfer protocols are important in modern networking. The goal of this lab is to create a simple file transfer system using the Transmission Control Protocol (TCP). Using Python’s socket library, we built a client-server application where the client sends a file to the server over a TCP/IP connection.

\section{Objectives}
The objectives of this lab were:
\begin{itemize}
    \item Implement a file transfer system using TCP sockets.
    \item Understand client-server communication using socket programming.
    \item Explore how data is transferred in chunks over a network.
\end{itemize}

\section{Methodology}
In this lab, the client-server architecture was implemented for file transfer over TCP/IP. The methodology was as follows:

\begin{enumerate}
    \item The **server** was created to listen for incoming client connections on a specific port.
    \item Once a client connected, the **server** received the file sent by the client in chunks, ensuring that large files could be handled efficiently.
    \item The **client** connected to the server, sent the filename, and then transferred the file data in 1024-byte chunks.
    \item The **server** saved the file to disk once the entire file had been received.
    \item Both the client and server closed the connection once the transfer was completed.
\end{enumerate}

\section{System Design}
The system has two main components: the **server** and the **client**. The server listens on a predefined port, accepts incoming connections, and receives files from the client. The client connects to the server, sends the file, and closes the connection once the transfer is complete.

\subsection{File Transfer Protocol}
The file transfer protocol follows a simple sequence:
\begin{enumerate}
    \item The client sends the filename to the server.
    \item The client then sends the file in 1024-byte chunks.
    \item The server receives each chunk and writes it to a file on the disk.
    \item Once the file transfer is complete, the server confirms the transfer, and both the client and server close the connection.
\end{enumerate}

\section{Implementation}
The implementation of the file transfer system was done using Python's `socket` library. The code for the server and client is provided below.

\subsection{Server Code}

\begin{lstlisting}[language=Python, caption=Server Code]
import socket

HOST = '172.19.219.4'  # Localhost
PORT = 12344        # Port to listen on

# Create a socket 
server_socket = socket.socket(socket.AF_INET, socket.SOCK_STREAM)
server_socket.bind((HOST, PORT))

# Start listening 
server_socket.listen(1)
print(f"Server listening on {HOST}:{PORT}...")

# Connection from client
conn, addr = server_socket.accept()
print(f"Connected by {addr}")

# Receive the file
file_name = conn.recv(1024).decode('utf-8')  # Receive file name
print(f"Receiving file: {file_name}")


with open(file_name, 'wb') as f:
    while True:
        
        data = conn.recv(1024)
        if not data:
            break  
        f.write(data)

print(f"File {file_name} received successfully.")
conn.close()  # Close 
server_socket.close()  
\end{lstlisting}

\subsection{Client Code}

\begin{lstlisting}[language=Python, caption=Client Code]
import socket

def client_program():
    # server and client must have the same IP and port
    host = '172.19.219.4'  
    port = 12344  # Port
    
    # TCP socket
    client_socket = socket.socket(socket.AF_INET, socket.SOCK_STREAM)
    
    # Connect to server
    client_socket.connect((host, port))
    
    # File
    filename = 'received_transfer_file.txt'  # Name the the file 
    
    # Try to receive the file from the server
    with open(filename, 'wb') as f:
        print(f"Receiving file from the server...")
        
        while True:
            data = client_socket.recv(1024)
            if not data:
                print("No more data received, closing connection.")
                break  # No more data received, end of file transfer
            f.write(data)  # Write data to the file

    print(f"File received successfully! Saved as {filename}")
    
    # Close socket connection
    client_socket.close()

if __name__ == '__main__':
    client_program()
\end{lstlisting}

\subsection{Sample Output from the Client}

\begin{verbatim}
Server listening on 172.19.219.4:12344...
Connected by ('172.19.208.1', 59923)
Receiving file: example.txt
File example.txt received successfully.
\end{verbatim}

The file `example.txt` was successfully transferred from the client to the server, and the server saved it in the current directory.

\section{Discussion}
This lab demonstrated the basic principles of file transfer over a TCP/IP connection using socket programming. The system successfully transferred files by breaking them into smaller chunks to handle larger files. 

Additionally, the client-server system could be upgraded to handle multiple file transfers simultaneously.

\section{Conclusion}
This lab shown how to use TCP/IP sockets to transfer files between a client and a server. By following the basic principles of socket programming, we were able to successfully implement a file transfer system that efficiently sends and receives files. 

\end{document}
